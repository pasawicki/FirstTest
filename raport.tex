\documentclass[11pt, onecolumn, twoside]{report}
\usepackage{polski}
\usepackage[utf8]{inputenc}
\usepackage{graphics}
\usepackage{fancyhdr}

\title{Teoria gier}
\author{Paweł Sawicki}
\date{\today}

\begin{document}

\pagestyle{fancy}
\fancyfoot[C]{}
\fancyfoot[R]{\thepage}
\fancyhead[L]{}
\fancyhead[C]{Teoria gier}
\fancyhead[R]{}

\begin{titlepage}
\maketitle
\end{titlepage}

\newpage
\begin{abstract}
W tym raporcie opowiemy o teorii gier. Teoria gier to dziedzina matematyki służąca do badania konfliktów interesów. Przy użyciu wywodzących się z niej metod obliczeniowych można określić właściwy lub przewidywany podział kosztów, jak również wytłumaczyć liczne zjawiska społeczne, na przykład nadmierną eksploatację łowisk morskich.
\end{abstract}
\tableofcontents
\newpage
\chapter{Koncert Jankiela na cymbałach}
Rozdział pierwszy umożliwi rozważenie implikacji wspaniałego koncertu na~cymbałach, jaki dał Jankiel w~Soplicowie wiosną 1812~roku.
\newpage
\section{Cymbał jako archetyp patrioty polskiego}
Jankiel, jak skądinąd~\cite{bibosz1} wiadomo, ,,kazał do siebie ze~wsi przychodzić muzyce, przy~której i~basetla była, i~kozice". Basetla to rodzaj instrumentu strunoewgo, uzyskuje~się ją poprzez hodowlę selektywną bassetów jodłujących na nutę ,,la". Kozic również można używać do~generacji melodii, największą rolę odgrywa wówczas tresura licówki, czyli samicy przewodzącej kierdlowi -- stadu kozic~\cite{bibosz3}. Wydaje się, że dzięki podobnej zasadzie Jankiel był w stanie zagrać koncert na cymbałach, mianowicie poprzez ustawienie cymbałów rzędem w~kolejności basowo-tenorowej i~rytmiczne uderzanie ich w~głowy batutą, czyli odmianą ciężkiej~\cite{bibosz2} pały dębowej (średnia długość 2,5 metra, średnia masa 75 kilogramów~\cite{bibosz1}). Cymbałowie ci byli niewątpliwie patriotami. Nie ustalono dotychczas, w~jaki sposób artysta uniemożliwił im oddalenie się z~miejsca koncertu.
\newpage
\section{Jankiel - obraz literacki Chopina?}
Jankiela można uważać lub nie uważać za obraz literacki Chopina, dopóki jedna z tych możliwości nie zostanie umieszczona w Kodeksie karnym. Prace trwają~\cite{bibosz2}. Więcej na ten temat w rozdziale~\ref{sw}.
\newpage

\chapter{Świnica}
\label{sw}

\section{Świnica w czasach Macieja Sieczki}

\section{Świnica dziś i jutro}

\newpage
\begin{thebibliography}{3}
\bibitem{bibosz1}
K. I. Chavsky; K. A. Shelletz; W. Yssypkah, \textit{Pandemies in fantasy literature}, London Contagious Society, 2015
\bibitem{bibosz2}
Z. Zwężony; W. Wątpliwy; W. Wielbłąd, \textit{Monografia uch igielnych}, Klub Robótek Domowych, 2018
\bibitem{bibosz3}
L. Stur, \textit{Este jedle rasti na Krivanskej stranie}, nakład własny, 1845
\end{thebibliography}

\end{document}